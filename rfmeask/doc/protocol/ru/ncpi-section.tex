\section{Система доставки уведомлений NCPI.}

\setlength{\parskip}{0em}
\setlength{\parindent}{0mm}
\setlength{\leftskip}{8mm}

\subsection{Введение.}
Система NCPI позволяет получать информацию о событиях, без необходимости постоянного опроса состояния устройств в режиме уведомлений. 

После подключения к серверу клиент должен активировать необходимые ему темы уведомлений с помощью команд описанных в настоящем разделе. Одновременно клиент может подписаться на любое количество тем, но не более одного раза на одну тему. По активации подписки, при возникновении события, сервер будет генерировать сообщения уведомлений в формате "Имя:Темы [Ответ]". Использование протокола TCP даёт гарантию доставки сообщений.

Система NCPI имеет три типа сообщений:

\setlength{\parskip}{0em}
\setlength{\parindent}{0mm}
\setlength{\leftskip}{8mm}

\begin{itemize} 
\item Простое событие.
\item Изменение состояния, событие с аргументом
\item Уведомление о непрерывных параметрах.
\end{itemize}

\subsubsection*{Простое событие}
Простое событие предполагает уведомление формата: SUBSYST:THEME  и не несет никакой дополнительной информации, кроме факта возникновения события. \\
\begin{gverb}
NOT:SUBSYST:THEME 1         % Активировать подписку
NOT:SUBSYST:THEME 0         % Отменить подписку
                            %
> SUBSYST:THEME             % Формат уведомления
\end{gverb}

\subsubsection*{Уведомление об изменении состояния, событии с аргументом}
Уведомление об изменении состояния возникает при изменении соответствующей переменной и передаёт связанную с ней строку, код или значение. \\
\begin{gverb}
NOT:SUBSYST:THEME 1         % Активировать подписку
NOT:SUBSYST:THEME 0         % Отменить подписку
                            %
> SUBSYST:THEME <argument>  % Формат уведомления
\end{gverb}

\subsubsection*{Уведомления о изменении непрерывного параметра.}
Данный тип предназначены для получения информации о непрерывных параметрах, изменяющихся с течением времени (например, текущее положение). Поддерживаются две стратегии получения уведомлений:

\setlength{\parskip}{1em}
1. Уведомление с временным дискретом, выраженным в миллисекундах. Дискрет задаёт минимальное временной промежуток между уведомлениями. Система может высылать сообщения реже этого интервала. Кроме того существует минимальный интервал времени чаще которого сообщения не могут быть посланы (обычно этот интервал составляет несколько десятков миллисекунд). 
\begin{gverb}
NOT:SUBSYST:THEME TIMERED,1000  % Уведомления каждую секунду.
NOT:SUBSYST:THEME 0             % Отменить подписку
                                %
SUBSYST:THEME <argument>        % Формат уведомления
\end{gverb}

2. Уведомление при изменении параметра на заданное значение.
Как и в случае с уведомлениями по истечению интервалов времени и итервал может превышать установленное значение.
\begin{gverb}
NOT:SUBSYST:THEME SMOOTH,5.34   % Уведомления с дискретом величин 5.34.
NOT:SUBSYST:THEME 0             % Отменить подписку
                                %
SUBSYST:THEME <argument>        % Формат уведомления
\end{gverb}

\setlength{\parskip}{0em}

\subsection{Пример}
\begin{gverb}
NOT:AXIS2:OPSTAT 1   % Уведомления на изменение состояния оси 2
                     %
> AXIS2:OPSTAT 1     % Ось 2 начала движение
> AXIS2:OPSTAT 0     % Ось 2 завершила движение
                     %
NOT:AXIS2:OPSTAT 0   % Отменить подписку
\end{gverb}


\setlength{\parskip}{0em}
\setlength{\parindent}{0em}
\setlength{\leftskip}{8mm}

\subsection{Перечень комманд NCPI}
\setlength{\parskip}{0em}

\subsubsection*{NOT:SYSTem:STATus}
	Уведомление о готовности системы. \\
	Тип: изменение состояния. \\ 

	Система считается готовой, если ни одно из устройств не сигнализирует об ошибке. \\
	Параметр:\\
	0 – Готов\\
	1 – Не готов

\subsubsection*{NOT:AXIS\#:STATus}
	Уведомление о готовности оси.\\ 
	Тип: изменение состояния.\\
	
	Ось считается готовой, если ни одно из устройств оси не сигнализирует об ошибке. \\
	Параметр:\\
	0 – Готов\\
	1 – Не готов

\subsubsection*{NOT:DEV\#:STATus}
	Уведомление о готовности устройства. \\
	Тип: изменение состояния\\

	Параметр:\\
	0 – Готов\\
	1 – Не готов

\subsubsection*{NOT:AXIS\#:POSition}
	Уведомление о текущем положении по оси.\\
	Тип: непрерывный параметр 

\subsubsection*{NOT:AXIS\#:UPOSition}
	Уведомление о положении по оси в единицах unit.\\
	Тип: непрерывный параметр 

\subsubsection*{NOT:AXIS\#:OPSTATus}
	Уведомление о статусе операции.\\
	Тип: изменение состояния\\

	Параметр:\\
	0 – Нет операции\\
	1 – Задание на движение\\
	2 – Режим сканирования

\subsubsection*{NOT:AXIS\#:OPSTOPtype}
	Уведомление о типе прерывания последней операции.\\
	Тип: изменение состояния\\	

	Параметр:\\
	0 – операция начата\\
	1 – нормальное завершение операции\\
	2 – операция завершена командой стоп\\
	3 – аварийное завершение операции
	
\subsubsection*{NOT:AXIS\#:SCAN:POINT}
	Уведомление о достижении новой точки в режиме сканирования. \\
	Тип: событие с аргументом\\

	Возвращает номер точки в теле уведомления.
	
\subsubsection*{NOT:AXIS\#:SCAN:TRIGERROR}
	Уведомление об ошибки в системе синхронизации при ожидании обратного триггера. Момент генерации следующего триггера настал ранее, чем вернулся обратный.\\
    Тип: событие\\

\subsubsection*{NOT:AXIS\#:SCAN:LSWItch}
	Уведомление об изменении статуса концевых датчиков.\\
	Тип: изменение состояния\\

	Параметр:\\
	0 – нет срабатывания\\
	1 - левый\\
	2 - правый\\
	10 – выдается при срабатывании обоих концевиков (вероятна ошибка)
