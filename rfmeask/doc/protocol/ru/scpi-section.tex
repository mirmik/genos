\section{SCPI комманды.}
\setlength{\parskip}{0em}
\setlength{\parindent}{0em}
\setlength{\leftskip}{8mm}

\subsection{Протокол и стек ошибок.}
Запрос установочных данных о версии протокола и состоянии стека ошибок. Управление стеком ошибок. 
\subsubsection*{SYSTem:VERSion?}
	Версия SCPI протокола.
\subsubsection*{SYSTem:ERRor[:NEXT]?}
	Возврат следующего сообщения об ошибке.
\subsubsection*{SYSTem:ERRor:COUNt?}
	Количество ошибок в стеке ошибок.
\subsubsection*{*CLS} 
	Очистка стека ошибок.

\subsection{Прочие стандартные функции.}
\subsubsection*{*ESE}
	Игнорируется.
\subsubsection*{*ESE?}	
	Игнорируется. (возврат единицы)
\subsubsection*{*ESR?}
	Игнорируется. (возврат единицы)
\subsubsection*{*OPC}
	Игнорируется.
\subsubsection*{*OPC?}
	Игнорируется. (возврат единицы)
\subsubsection*{*RST}
	Игнорируется.
\subsubsection*{*SRE}
	Игнорируется.
\subsubsection*{*SRE?}
	Игнорируется. (возврат единицы)
\subsubsection*{*STB?}
	Игнорируется. (возврат единицы)
\subsubsection*{*WAI}
	Игнорируется.


\subsection{Информация об изделии.}
Запрос неизменяемых параметров прибора и осей изделия.

\subsubsection*{*IDN?}
	Запрос идентификатора изделия. 

\subsubsection*{SYSTem:DEVSTOTal?}
	Количество зарегистрированных устройств.

\subsubsection*{SYSTem:axes_total?}
	Количество зарегистрированных осей.


\subsection{Системные команды.}
Подмножество комманд, влияющих на все оси и устройства в системе, cостояние системы в целом.

\subsubsection*{SYSTem:PRESet}
	Сброс всех устройств в состояние по умолчанию, сброс всех настроек.

\subsubsection*{SYSTem:IPADDR}
	Установка IPv4 адреса изделия. В записи адреса вместо символа точки следует использовать символы запятых.  
	Пример:
	\begin{gverb}
	SYST:IPADDR 192,168,1,42
	\end{gverb}

\subsubsection*{SYSTem:STOP}
	Выполнить остановку движения по всем осям.

\subsubsection*{SYSTem:POWOFF}
	Выполнить отключения питания исполнительных устройств на всех осях.\\
	\emph{* Только для отдельных изделий. Для большинства изделий управление питанием не требуется.}

\subsubsection*{SYSTem:STATus?}
	Запрос статуса готовности системы\\
	Система возвращает готовность если все входящие 
	в её состав устройства и оси возвращают готовность.
	\\
	Ответ:\\
	0 – Система готова.\\
	1 – Система не готова.\\

\subsection{Состояние устройств.}
Информация о состоянии аппаратных устройств.

\subsubsection*{DEV\#:IDN?}
	Запрос идентификатора устройства.

\subsubsection*{DEV\#:PRESET}
	Возврат параметров устройства к параметрам по умолчанию.

\subsubsection*{DEV\#:STATus?}
	Запрос статуса готовности устройства.\\
	Ответ:\\
	0 – Устройство готово.\\
	1 – Устройство не готово.

\subsubsection*{DEV\#:ALM?}
	Запрос alarm-кода устройства в случае его неисправности.
	\emph{* Формат ответа зависит от типа устройства.}

\subsection{Управление осями. Совместимость.}
В зависимости от типа сервопривода и функционального назначения осей, ось может поддерживать дополнительные функции.
 
\subsubsection*{AXIS\#:COMPat:SCAN?}
	Запрос наличия на оси подсистемы синхронного сканирования.

\subsubsection*{AXIS\#:COMPat:REFSet?}
	Запрос наличия функции установки текущего абсолютного положения (см. AXIS\#:SETREF, AXIS\#:SETUREF).

\subsection{Управление осями. Информация и состояние.}
\subsubsection*{AXIS\#:STATus:IDN?}
	Запрос идентификатора оси.

\subsubsection*{AXIS\#:STATus:DEVS?}
	Запрос кодов устройств, входящих в ось.

\subsubsection*{AXIS\#:STATus:POSition?}
	Запрос положения в импульсах энкодера.

\subsubsection*{AXIS\#:STATus:UPOSition?}
	Запрос положения в юнитах.

\subsubsection*{AXIS\#:SETTINGS:RATIO?}
	Количество импульсов энкодера в одном юните.

\subsubsection*{AXIS\#:SETTINGS:DEFSPEed?}
	Скорость по умолчанию в оборотах в минуту.

\subsubsection*{AXIS\#:SETTINGS:DEFACCel?}
	Длительность разгона по умолчанию в миллисекундах.

\subsubsection*{AXIS\#:SETTINGS:MAXSPEed?}
	Максимальная скорость в оборотах в минуту.

\subsubsection*{AXIS\#:SETTINGS:MINAccel?}
	Минимальное временя разгона в миллисекундах.

\subsubsection*{AXIS\#:STATus[:STATus]?}
	Запрос статуса готовности оси.\\
	\\
	Ответ:\\
	0 – Ось готова.\\
	1 – Ось не готова.

\subsubsection*{AXIS\#:STATus:LSWItch?}
	Запрос статуса концевых датчиков. \\
	\\
	Ответ:\\
	0 – нет срабатывания,\\
	1 - левый,\\
	2 - правый,\\
	10 – выдается при срабатывании обоих концевиков (вероятна ошибка).
	\emph{* Только для изделий с концевыми датчиками. При отсутствии таковых всегда возвращает 0}

\subsubsection*{AXIS\#:STATus:OPcode?}
	Запрос кода выполняемой операции.\\
	\\
	Ответ:\\
	0 - Нет операции,\\
	1 -  Выполняется перемещение,\\
	-1 - Ось в состоянии инициализации\\

\subsection{Управления осями. Системные функции.}
\subsubsection*{AXIS\#:PRESET}
	Возврат устройств оси к значениям по умолчанию.

\subsubsection*{AXIS\#:SON}
	[Снять тормоз] и подать питание на двигатель.\\
	\emph{* Только для отдельных изделий. Для большинства изделий управление питанием не требуется.}

\subsubsection*{AXIS\#:SOFF}
	Снять питание с двигателя, [установить тормоз].\\
	\emph{* Только для отдельных изделий. Для большинства изделий управление питанием не требуется.}

\subsection{Управления осями. Програмные пределы, установка нулей.}
В отличии от других настроек, настройки нулей и пределов перемещения не сбрасываются при отключении питания. 

\subsubsection*{AXIS\#:SETTINGS:UBACKLIMit <arg>}
\subsubsection*{AXIS\#:SETTINGS:UBACKLIMit?}
	Задать/считать предел перемещения при движении в отрицательном направлении.

\subsubsection*{AXIS\#:SETTINGS:UFORWLIMit <arg>}
\subsubsection*{AXIS\#:SETTINGS:UFORWLIMit?}
	Задать/считать предел перемещения при движении в положительном направлении.

\subsubsection*{AXIS\#:SETTINGS:ULIMITS <back>, <forw>}
	Задать/считать пару пределов перемещения. back должен быть меньше forw.

\subsubsection*{AXIS\#:SETZERo}
	Сброс абсолютного положения в ноль.

\subsubsection*{AXIS\#:SETREFerence <pos>}
	Сброс абсолютного положения в заданное значение. \\
	pos – новая текущая позиция в импульсах.\\
	\emph{* Поддерживается не для всех типов сервоусилителей. см. AXIS\#:COMPat:REFSet?}

\subsubsection*{AXIS\#:SETUREFerence <upos>}
	Сброс абсолютного положения в заданное значение (в юнитах). \\
	upos– новая текущая позиция в юнитах.\\
	\emph{* Поддерживается не для всех типов сервоусилителей. см. AXIS\#:COMPat:REFSet?}

\subsection{Управления осями. Движение.}

\subsubsection*{AXIS\#:SPEeed <rpm>}
\subsubsection*{AXIS\#:SPEeed?}
	Установка и запрос скорости в об/мин.

\subsubsection*{AXIS\#:USPEeed <ups>}
\subsubsection*{AXIS\#:USPEeed?}
	Установка и запрос скорости в unit/сек.

\subsubsection*{AXIS\#:ACCel <ms>}
\subsubsection*{AXIS\#:ACCel?}
	Установка и запрос интервала времени разгона и торможения для операций типа MOV и JOG. Интервал времени выражен в миллисекундах. 
	ms – интервал времени в миллисекундах. 

\subsubsection*{AXIS\#:MOVe[:RELative] <distance>}
	Относительное перемещение (в импульсах энкодера).\\
	distance – количество импульсов. Знак определяет направление перемещения.

\subsubsection*{AXIS\#:MOVe:ABSolute <position>}
	Абсолютное перемещение (в импульсах энкодера).\\
	position – координата в импульсах.

\subsubsection*{AXIS\#:UMOVe[:RELative] <distance>}
	Относительное перемещение (в unit).\\
	distance – расстояние в юнитах. Знак определяет направление перемещения.

\subsubsection*{AXIS\#:UMOVe:ABSolute <position>}
	Абсолютное перемещение (в unit).\\
	position – координата в юнитах.

\subsubsection*{AXIS\#:JOG <direction>}
	Движение с постоянной скоростью. \\
	direction – 1 – вперёд, -1 - назад.

\subsubsection*{AXIS\#:STOP}
	Остановить ось.

\subsubsection*{AXIS\#:UNSAFE:MOVe <distance>}
	Относительное перемещение (в импульсах энкодера) с игнорированием пределов.\\
	Данная команда может быть использована при наладке оборудования. Не рекомендуется для использования в штатном режиме. 

\subsubsection*{AXIS\#:UNSAFE:UMOVe <distance>}
	Относительное перемещение (в юнитах) с игнорированием пределов.\\
	Данная команда может быть использована при наладке оборудования. Не рекомендуется для использования в штатном режиме. 

\subsection{Управление осями. Непрерывное сканирование}
Функции системы непрерывного сканирования доступны для осей, на которых установлен модуль синхронизации.
При активации режима непрерывного сканирования устройство синхронизации будет посылать триггерные сигналы при проходе через установленные точки. 

Также при проходе через установленные точки или после получения обратного триггера (см. NOTRIGMODE)) генерируются уведомления (см. NOT:AXIS\#:SCANPOINT). Для получения информации о проходе очередной точки на стороне клиента необходимо подключить соответствующее уведомление. \\

Шаг сканирования вычисляется по формуле: \\STEP = UMOVE / (POINTS - 1)

\subsubsection*{AXIS\#:SCAN:MOVE <distance>}
\subsubsection*{AXIS\#:SCAN:MOVE?}
\subsubsection*{AXIS\#:SCAN:UMOVE <distance>}
\subsubsection*{AXIS\#:SCAN:UMOVE?}
	Установка зоны сканирования.\\
	distance – расстояние, обрабатываемое алгоритмом сканирования в импульсах/юнитах 
	Функции с префиксом U здесь и далее - аналоги в юнитах.

\subsubsection*{AXIS\#:SCAN:FWRDzone <distance>}
\subsubsection*{AXIS\#:SCAN:FWRDzone?}
\subsubsection*{AXIS\#:SCAN:UFWRDzone <distance>}
\subsubsection*{AXIS\#:SCAN:UFWRDzone?}
	Установка расстояния до первой точки.\\
	distance – расстояние, пропускаемое алгоритмом перед началом сканирования в импульсах/юнитах.

\subsubsection*{AXIS\#:SCAN:BWRDzone <distance>}
\subsubsection*{AXIS\#:SCAN:BWRDzone?}
\subsubsection*{AXIS\#:SCAN:UBWRDzone <distance>}
\subsubsection*{AXIS\#:SCAN:UBWRDzone?}
	Установка расстояния прохода после последней точки (только для режима SCAN:START).\\
	distance – расстояние в импульсах/юнитах.
	
\subsubsection*{AXIS\#:SCAN:COMPSTART}
	Активировать режим сканирования. 
	Активации сканирования не приводит к началу движения. Данная команда включает режим выработки триггеров и передачи уведомлений о достижении точек.

\subsubsection*{AXIS\#:SCAN:POINTS <npnts>}
\subsubsection*{AXIS\#:SCAN:POINTS?}
	Установка количества точек сканирования.\\
	npnts – количество обходимых точек.

\subsubsection*{AXIS\#:SCAN:NOTRIGMODE <en>}
	Включить/отключить ожидание обратного тригера перед выдачей уведомления о достижении очередной точки.\\
	en:\\
	0 - уведомление о достижении очередной точке будет послано после прихода обратного триггера (режим по умолчанию)\\
	1 - ожидание обратного триггера отключено\\
	
\subsubsection*{AXIS\#:MANTRIGmode <en>}
	Перевести ось в режим ручной генерации триггеров.\\
	Настраивает синхронизатор в режим работы с командой AXIS\#:TRIGGER.\\
	en:\\
	1 - включить ручной режим\\
	0 - выключить ручной режим 

\subsubsection*{AXIS\#:TRIGGER}
	Сгенерировать триггер.\\
	Перед использованием следует первести ось в режим ручной генерации триггеров (см. AXIS\#:MANTRIGmode).

\subsubsection*{AXIS\#:TRIGRETTIME?}
	Запрос времени возврата обратного триггера в миллисекундах.