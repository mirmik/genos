\documentclass{article}
\usepackage[left=40mm, top=30mm, right=30mm, bottom=30mm]{geometry}

\usepackage{multicol}
\usepackage{lipsum}
\usepackage{titlepic}
\usepackage{graphicx}
\usepackage{titling}
\usepackage{titlesec}

\usepackage[utf8x]{inputenc}
\usepackage[T2A]{fontenc}
\usepackage[english]{babel}

\usepackage{fancyvrb,newverbs,xcolor}
\usepackage{verbatimbox}
\usepackage{fancybox,fancyhdr}
\usepackage{lipsum}
\usepackage{listings}
\usepackage{lastpage}

\lstset{inputencoding=utf8x, extendedchars=\true}
\setlength{\parindent}{5ex}

\definecolor{mygreen}{rgb}{0,0.6,0}
\definecolor{myblue}{rgb}{0,0.2,0.9}
\definecolor{mybrown}{rgb}{0.73,0.47,0.37}
\definecolor{mypurple}{rgb}{0.4,0.1,0.4}

\lstdefinestyle{python}{
  language=Python,
  showstringspaces=false,
  backgroundcolor=\color{white},   % choose the background color
  basicstyle=\small\ttfamily,        % size of fonts used for the code
  breaklines=true,                 % automatic line breaking only at whitespace
  captionpos=b,                    % sets the caption-position to bottom
  commentstyle=\color{mygreen},    % comment style
  escapeinside={\%*}{*)},          % if you want to add LaTeX within your code
  keywordstyle=\color{myblue},       % keyword style
  stringstyle=\color{mybrown},     % string literal style
  emph={True, False},
  emphstyle={\color{mypurple}},
}


\pagestyle{fancy} 

\fancyfoot[R]{\thepage/\pageref{LastPage}}
\fancyfoot[L]{rev. 3.2}
\fancyfoot[C]{}

\renewcommand{\footrulewidth}{0.3 mm} 
\renewcommand{\headrulewidth}{0.3 mm}

\titlespacing{\subsubsection}{8mm}{5mm}{0mm}

\newenvironment{gverb}
{\verbbox}
{\endverbbox\par\colorbox{lightgray}{\parbox{0.9\textwidth}{\theverbbox}}\par}

\setlength{\parskip}{0em}
\setlength{\parindent}{0mm}
\setlength{\leftskip}{8mm}

\begin{document}

\pretitle{
  \begin{center}
  \includegraphics[width=6cm,height=6cm]{smitek.jpg}\\[\bigskipamount]
}
\posttitle{\end{center}}

\date{May 26, 2020}
\title{\Large{Positioner's API Reference Manual}}
\author{Radioline}

\clearpage
\maketitle
\thispagestyle{empty}
\newpage

\tableofcontents
\newpage
\setlength{\parskip}{1em}

\section{Introduction}

This document describes the command system used for manual and automatic positioner control.

The main logical object of the protocol is logical axis. 
Logical axis represents one or more devices such as servo amplifier or syncronization module.
All commands are divided into axis commands, device commands and system commands.
Commands in this manual are sorted by functionality, their description, 
the specifics of application are noted. 
Usage examples are given where necessary.

Manual provides general description of control commands.
Depending on specific configuration, positioner type or functional 
purpose of its logical axes some commands are not implemented or are ignored.
Please check with the manufacturer for specific product type features.

\subsection{Protocol}
Positioner is controlled by the SCPI std V1999.0 4.2.1 protocol extended by a notification subscription system (NCPI).

SCPI protocol is used for parameters setup, devices status queries, move commands. 
There are three types of commands:
\begin{itemize}
  \item\textbf{:SYSTem:} -- system commands,
  \item\textbf{:AXIS\#:} -- logical axes functions,
  \item\textbf{:DEV\#:} -- hardware modules functions (serve amplifiers and other devices).
\end{itemize}

NCPI extension allows you to receive information about positioner state change in form of notifications.
The only type here is:
\begin{itemize}
  \item\textbf{:NOT:} - notification subscription.
\end{itemize}

Communication with positioner is carried out via TCP/IP through the rj45 (Ethernet) сonnector on the controller.
Two tcp ports are used:
\begin{itemize}
  \item SCPI console port 5025 (standard port for SCPI consoles),
  \item NCPI console port 5026.
\end{itemize}

When working with positioner via VISA library, socket mode is supported, instrument mode is not.

\newpage

\subsection{Namespacing}

\textbf{COMMAND\#} -- where '\#' is a natural number (beginning with 0). 

\textbf{<argument>} -- the argument of a command is in <> brackets.

\textbf{CMD:[SUBCMD]} -- the part in [ ] brackets is an unnescesarry argument or part of a command. 

There are also code blocks:
\begin{gverb}
*IDN?                       % Command query.
> RADIOLINE,XXX,XXX,XXX     % The answer is indicated by symbol '>'.
                            % - oneline commentary.
\end{gverb}

\subsection{On distance units}

There are several types of distance units used in protocol commands:

\begin{itemize}
  \item Revolution -- an engine revolution.
  \item Pulse -- servo drive feedback discrete. 
        There is a constant ratio of the number of pulses to the number of revolutions.
  \item Unit -- natural unit based on system requirements 
        (typically a centimeter for linear axes and a degree for axes of rotation).
\end{itemize}

\emph{* Pulse number per unit for the axis can be requested \\ by a command "AXIS\#:SETTINGS:RATIO?".}

\subsection{On acceleration units (acceleration time)}

Acceleration is a number of milliseconds required for engine to reach its nominal operating mode. 
Please note that when the acceleration time is reduced, the acceleration increases. 

\newpage
\section{Getting started}

SCPI protocol allows you to work in automatic and manual modes, depending on the source of commands.
For automatic mode the source of commands is the operating software.
For manual mode commands are entered by user via the I/O terminal.
These modes have no functional differences and both work through established TCP connection to the SCPI port (5025 by default) of the controller.

Manual mode can be used for connection check, for control commands debugging 
before using them in automatic mode, 
also for simple commands execution when operating software is unavailable.

Before you start working with the protocol, you should check the availability of the 
controller in local network and establish connection to the SCPI port of the positioner. 
To establish connection in manual mode you can use such utilities as \emph{PuTTY} (mode: \emph{raw}) or analogs.

SCPI protocol packet is an ASCII string, terminated by a symbol of line ending (LF: $\backslash$n, dec:10, bin:0x0A).

Test product ID query:
\begin{gverb}
*IDN?
> RADIOLINE,XXX,XXX,XXX
\end{gverb}

Status query:
\begin{gverb}
SYST:STAT?              % Positioner ready state query.
> 0                     % Positioner is ready for work.
\end{gverb}

Axis info query: 
\begin{gverb}
SYST:AXESTOT?           % Number of axes query.
> 3                     % There are three axes in the system: 0, 1, 2.
AXIS2:UPOSition?        % Indexes are zero-based.
> 0.378123              
\end{gverb}

Parameters setup:
\begin{gverb}
AXIS2:USPD 2            % Parameters setup and executional commands
                        % are not followed by the answer.        
\end{gverb}

\newpage
\section{SCPI комманды.}
\setlength{\parskip}{0em}
\setlength{\parindent}{0em}
\setlength{\leftskip}{8mm}

\subsection{Протокол и стек ошибок.}
Запрос установочных данных о версии протокола и состоянии стека ошибок. Управление стеком ошибок. 
\subsubsection*{SYSTem:VERSion?}
	Версия SCPI протокола.
\subsubsection*{SYSTem:ERRor[:NEXT]?}
	Возврат следующего сообщения об ошибке.
\subsubsection*{SYSTem:ERRor:COUNt?}
	Количество ошибок в стеке ошибок.
\subsubsection*{*CLS} 
	Очистка стека ошибок.

\subsection{Прочие стандартные функции.}
\subsubsection*{*ESE}
	Игнорируется.
\subsubsection*{*ESE?}	
	Игнорируется. (возврат единицы)
\subsubsection*{*ESR?}
	Игнорируется. (возврат единицы)
\subsubsection*{*OPC}
	Игнорируется.
\subsubsection*{*OPC?}
	Игнорируется. (возврат единицы)
\subsubsection*{*RST}
	Игнорируется.
\subsubsection*{*SRE}
	Игнорируется.
\subsubsection*{*SRE?}
	Игнорируется. (возврат единицы)
\subsubsection*{*STB?}
	Игнорируется. (возврат единицы)
\subsubsection*{*WAI}
	Игнорируется.


\subsection{Информация об изделии.}
Запрос неизменяемых параметров прибора и осей изделия.

\subsubsection*{*IDN?}
	Запрос идентификатора изделия. 

\subsubsection*{SYSTem:DEVSTOTal?}
	Количество зарегистрированных устройств.

\subsubsection*{SYSTem:axes_total?}
	Количество зарегистрированных осей.


\subsection{Системные команды.}
Подмножество комманд, влияющих на все оси и устройства в системе, cостояние системы в целом.

\subsubsection*{SYSTem:PRESet}
	Сброс всех устройств в состояние по умолчанию, сброс всех настроек.

\subsubsection*{SYSTem:IPADDR}
	Установка IPv4 адреса изделия. В записи адреса вместо символа точки следует использовать символы запятых.  
	Пример:
	\begin{gverb}
	SYST:IPADDR 192,168,1,42
	\end{gverb}

\subsubsection*{SYSTem:STOP}
	Выполнить остановку движения по всем осям.

\subsubsection*{SYSTem:POWOFF}
	Выполнить отключения питания исполнительных устройств на всех осях.\\
	\emph{* Только для отдельных изделий. Для большинства изделий управление питанием не требуется.}

\subsubsection*{SYSTem:STATus?}
	Запрос статуса готовности системы\\
	Система возвращает готовность если все входящие 
	в её состав устройства и оси возвращают готовность.
	\\
	Ответ:\\
	0 – Система готова.\\
	1 – Система не готова.\\

\subsection{Состояние устройств.}
Информация о состоянии аппаратных устройств.

\subsubsection*{DEV\#:IDN?}
	Запрос идентификатора устройства.

\subsubsection*{DEV\#:PRESET}
	Возврат параметров устройства к параметрам по умолчанию.

\subsubsection*{DEV\#:STATus?}
	Запрос статуса готовности устройства.\\
	Ответ:\\
	0 – Устройство готово.\\
	1 – Устройство не готово.

\subsubsection*{DEV\#:ALM?}
	Запрос alarm-кода устройства в случае его неисправности.
	\emph{* Формат ответа зависит от типа устройства.}

\subsection{Управление осями. Совместимость.}
В зависимости от типа сервопривода и функционального назначения осей, ось может поддерживать дополнительные функции.
 
\subsubsection*{AXIS\#:COMPat:SCAN?}
	Запрос наличия на оси подсистемы синхронного сканирования.

\subsubsection*{AXIS\#:COMPat:REFSet?}
	Запрос наличия функции установки текущего абсолютного положения (см. AXIS\#:SETREF, AXIS\#:SETUREF).

\subsection{Управление осями. Информация и состояние.}
\subsubsection*{AXIS\#:STATus:IDN?}
	Запрос идентификатора оси.

\subsubsection*{AXIS\#:STATus:DEVS?}
	Запрос кодов устройств, входящих в ось.

\subsubsection*{AXIS\#:STATus:POSition?}
	Запрос положения в импульсах энкодера.

\subsubsection*{AXIS\#:STATus:UPOSition?}
	Запрос положения в юнитах.

\subsubsection*{AXIS\#:SETTINGS:RATIO?}
	Количество импульсов энкодера в одном юните.

\subsubsection*{AXIS\#:SETTINGS:DEFSPEed?}
	Скорость по умолчанию в оборотах в минуту.

\subsubsection*{AXIS\#:SETTINGS:DEFACCel?}
	Длительность разгона по умолчанию в миллисекундах.

\subsubsection*{AXIS\#:SETTINGS:MAXSPEed?}
	Максимальная скорость в оборотах в минуту.

\subsubsection*{AXIS\#:SETTINGS:MINAccel?}
	Минимальное временя разгона в миллисекундах.

\subsubsection*{AXIS\#:STATus[:STATus]?}
	Запрос статуса готовности оси.\\
	\\
	Ответ:\\
	0 – Ось готова.\\
	1 – Ось не готова.

\subsubsection*{AXIS\#:STATus:LSWItch?}
	Запрос статуса концевых датчиков. \\
	\\
	Ответ:\\
	0 – нет срабатывания,\\
	1 - левый,\\
	2 - правый,\\
	10 – выдается при срабатывании обоих концевиков (вероятна ошибка).
	\emph{* Только для изделий с концевыми датчиками. При отсутствии таковых всегда возвращает 0}

\subsubsection*{AXIS\#:STATus:OPcode?}
	Запрос кода выполняемой операции.\\
	\\
	Ответ:\\
	0 - Нет операции,\\
	1 -  Выполняется перемещение,\\
	-1 - Ось в состоянии инициализации\\

\subsection{Управления осями. Системные функции.}
\subsubsection*{AXIS\#:PRESET}
	Возврат устройств оси к значениям по умолчанию.

\subsubsection*{AXIS\#:SON}
	[Снять тормоз] и подать питание на двигатель.\\
	\emph{* Только для отдельных изделий. Для большинства изделий управление питанием не требуется.}

\subsubsection*{AXIS\#:SOFF}
	Снять питание с двигателя, [установить тормоз].\\
	\emph{* Только для отдельных изделий. Для большинства изделий управление питанием не требуется.}

\subsection{Управления осями. Програмные пределы, установка нулей.}
В отличии от других настроек, настройки нулей и пределов перемещения не сбрасываются при отключении питания. 

\subsubsection*{AXIS\#:SETTINGS:UBACKLIMit <arg>}
\subsubsection*{AXIS\#:SETTINGS:UBACKLIMit?}
	Задать/считать предел перемещения при движении в отрицательном направлении.

\subsubsection*{AXIS\#:SETTINGS:UFORWLIMit <arg>}
\subsubsection*{AXIS\#:SETTINGS:UFORWLIMit?}
	Задать/считать предел перемещения при движении в положительном направлении.

\subsubsection*{AXIS\#:SETTINGS:ULIMITS <back>, <forw>}
	Задать/считать пару пределов перемещения. back должен быть меньше forw.

\subsubsection*{AXIS\#:SETZERo}
	Сброс абсолютного положения в ноль.

\subsubsection*{AXIS\#:SETREFerence <pos>}
	Сброс абсолютного положения в заданное значение. \\
	pos – новая текущая позиция в импульсах.\\
	\emph{* Поддерживается не для всех типов сервоусилителей. см. AXIS\#:COMPat:REFSet?}

\subsubsection*{AXIS\#:SETUREFerence <upos>}
	Сброс абсолютного положения в заданное значение (в юнитах). \\
	upos– новая текущая позиция в юнитах.\\
	\emph{* Поддерживается не для всех типов сервоусилителей. см. AXIS\#:COMPat:REFSet?}

\subsection{Управления осями. Движение.}

\subsubsection*{AXIS\#:SPEeed <rpm>}
\subsubsection*{AXIS\#:SPEeed?}
	Установка и запрос скорости в об/мин.

\subsubsection*{AXIS\#:USPEeed <ups>}
\subsubsection*{AXIS\#:USPEeed?}
	Установка и запрос скорости в unit/сек.

\subsubsection*{AXIS\#:ACCel <ms>}
\subsubsection*{AXIS\#:ACCel?}
	Установка и запрос интервала времени разгона и торможения для операций типа MOV и JOG. Интервал времени выражен в миллисекундах. 
	ms – интервал времени в миллисекундах. 

\subsubsection*{AXIS\#:MOVe[:RELative] <distance>}
	Относительное перемещение (в импульсах энкодера).\\
	distance – количество импульсов. Знак определяет направление перемещения.

\subsubsection*{AXIS\#:MOVe:ABSolute <position>}
	Абсолютное перемещение (в импульсах энкодера).\\
	position – координата в импульсах.

\subsubsection*{AXIS\#:UMOVe[:RELative] <distance>}
	Относительное перемещение (в unit).\\
	distance – расстояние в юнитах. Знак определяет направление перемещения.

\subsubsection*{AXIS\#:UMOVe:ABSolute <position>}
	Абсолютное перемещение (в unit).\\
	position – координата в юнитах.

\subsubsection*{AXIS\#:JOG <direction>}
	Движение с постоянной скоростью. \\
	direction – 1 – вперёд, -1 - назад.

\subsubsection*{AXIS\#:STOP}
	Остановить ось.

\subsubsection*{AXIS\#:UNSAFE:MOVe <distance>}
	Относительное перемещение (в импульсах энкодера) с игнорированием пределов.\\
	Данная команда может быть использована при наладке оборудования. Не рекомендуется для использования в штатном режиме. 

\subsubsection*{AXIS\#:UNSAFE:UMOVe <distance>}
	Относительное перемещение (в юнитах) с игнорированием пределов.\\
	Данная команда может быть использована при наладке оборудования. Не рекомендуется для использования в штатном режиме. 

\subsection{Управление осями. Непрерывное сканирование}
Функции системы непрерывного сканирования доступны для осей, на которых установлен модуль синхронизации.
При активации режима непрерывного сканирования устройство синхронизации будет посылать триггерные сигналы при проходе через установленные точки. 

Также при проходе через установленные точки или после получения обратного триггера (см. NOTRIGMODE)) генерируются уведомления (см. NOT:AXIS\#:SCANPOINT). Для получения информации о проходе очередной точки на стороне клиента необходимо подключить соответствующее уведомление. \\

Шаг сканирования вычисляется по формуле: \\STEP = UMOVE / (POINTS - 1)

\subsubsection*{AXIS\#:SCAN:MOVE <distance>}
\subsubsection*{AXIS\#:SCAN:MOVE?}
\subsubsection*{AXIS\#:SCAN:UMOVE <distance>}
\subsubsection*{AXIS\#:SCAN:UMOVE?}
	Установка зоны сканирования.\\
	distance – расстояние, обрабатываемое алгоритмом сканирования в импульсах/юнитах 
	Функции с префиксом U здесь и далее - аналоги в юнитах.

\subsubsection*{AXIS\#:SCAN:FWRDzone <distance>}
\subsubsection*{AXIS\#:SCAN:FWRDzone?}
\subsubsection*{AXIS\#:SCAN:UFWRDzone <distance>}
\subsubsection*{AXIS\#:SCAN:UFWRDzone?}
	Установка расстояния до первой точки.\\
	distance – расстояние, пропускаемое алгоритмом перед началом сканирования в импульсах/юнитах.

\subsubsection*{AXIS\#:SCAN:BWRDzone <distance>}
\subsubsection*{AXIS\#:SCAN:BWRDzone?}
\subsubsection*{AXIS\#:SCAN:UBWRDzone <distance>}
\subsubsection*{AXIS\#:SCAN:UBWRDzone?}
	Установка расстояния прохода после последней точки (только для режима SCAN:START).\\
	distance – расстояние в импульсах/юнитах.
	
\subsubsection*{AXIS\#:SCAN:COMPSTART}
	Активировать режим сканирования. 
	Активации сканирования не приводит к началу движения. Данная команда включает режим выработки триггеров и передачи уведомлений о достижении точек.

\subsubsection*{AXIS\#:SCAN:POINTS <npnts>}
\subsubsection*{AXIS\#:SCAN:POINTS?}
	Установка количества точек сканирования.\\
	npnts – количество обходимых точек.

\subsubsection*{AXIS\#:SCAN:NOTRIGMODE <en>}
	Включить/отключить ожидание обратного тригера перед выдачей уведомления о достижении очередной точки.\\
	en:\\
	0 - уведомление о достижении очередной точке будет послано после прихода обратного триггера (режим по умолчанию)\\
	1 - ожидание обратного триггера отключено\\
	
\subsubsection*{AXIS\#:MANTRIGmode <en>}
	Перевести ось в режим ручной генерации триггеров.\\
	Настраивает синхронизатор в режим работы с командой AXIS\#:TRIGGER.\\
	en:\\
	1 - включить ручной режим\\
	0 - выключить ручной режим 

\subsubsection*{AXIS\#:TRIGGER}
	Сгенерировать триггер.\\
	Перед использованием следует первести ось в режим ручной генерации триггеров (см. AXIS\#:MANTRIGmode).

\subsubsection*{AXIS\#:TRIGRETTIME?}
	Запрос времени возврата обратного триггера в миллисекундах.

\newpage
\section{Система доставки уведомлений NCPI.}

\setlength{\parskip}{0em}
\setlength{\parindent}{0mm}
\setlength{\leftskip}{8mm}

\subsection{Введение.}
Система NCPI позволяет получать информацию о событиях, без необходимости постоянного опроса состояния устройств в режиме уведомлений. 

После подключения к серверу клиент должен активировать необходимые ему темы уведомлений с помощью команд описанных в настоящем разделе. Одновременно клиент может подписаться на любое количество тем, но не более одного раза на одну тему. По активации подписки, при возникновении события, сервер будет генерировать сообщения уведомлений в формате "Имя:Темы [Ответ]". Использование протокола TCP даёт гарантию доставки сообщений.

Система NCPI имеет три типа сообщений:

\setlength{\parskip}{0em}
\setlength{\parindent}{0mm}
\setlength{\leftskip}{8mm}

\begin{itemize} 
\item Простое событие.
\item Изменение состояния, событие с аргументом
\item Уведомление о непрерывных параметрах.
\end{itemize}

\subsubsection*{Простое событие}
Простое событие предполагает уведомление формата: SUBSYST:THEME  и не несет никакой дополнительной информации, кроме факта возникновения события. \\
\begin{gverb}
NOT:SUBSYST:THEME 1         % Активировать подписку
NOT:SUBSYST:THEME 0         % Отменить подписку
                            %
> SUBSYST:THEME             % Формат уведомления
\end{gverb}

\subsubsection*{Уведомление об изменении состояния, событии с аргументом}
Уведомление об изменении состояния возникает при изменении соответствующей переменной и передаёт связанную с ней строку, код или значение. \\
\begin{gverb}
NOT:SUBSYST:THEME 1         % Активировать подписку
NOT:SUBSYST:THEME 0         % Отменить подписку
                            %
> SUBSYST:THEME <argument>  % Формат уведомления
\end{gverb}

\subsubsection*{Уведомления о изменении непрерывного параметра.}
Данный тип предназначены для получения информации о непрерывных параметрах, изменяющихся с течением времени (например, текущее положение). Поддерживаются две стратегии получения уведомлений:

\setlength{\parskip}{1em}
1. Уведомление с временным дискретом, выраженным в миллисекундах. Дискрет задаёт минимальное временной промежуток между уведомлениями. Система может высылать сообщения реже этого интервала. Кроме того существует минимальный интервал времени чаще которого сообщения не могут быть посланы (обычно этот интервал составляет несколько десятков миллисекунд). 
\begin{gverb}
NOT:SUBSYST:THEME TIMERED,1000  % Уведомления каждую секунду.
NOT:SUBSYST:THEME 0             % Отменить подписку
                                %
SUBSYST:THEME <argument>        % Формат уведомления
\end{gverb}

2. Уведомление при изменении параметра на заданное значение.
Как и в случае с уведомлениями по истечению интервалов времени и итервал может превышать установленное значение.
\begin{gverb}
NOT:SUBSYST:THEME SMOOTH,5.34   % Уведомления с дискретом величин 5.34.
NOT:SUBSYST:THEME 0             % Отменить подписку
                                %
SUBSYST:THEME <argument>        % Формат уведомления
\end{gverb}

\setlength{\parskip}{0em}

\subsection{Пример}
\begin{gverb}
NOT:AXIS2:OPSTAT 1   % Уведомления на изменение состояния оси 2
                     %
> AXIS2:OPSTAT 1     % Ось 2 начала движение
> AXIS2:OPSTAT 0     % Ось 2 завершила движение
                     %
NOT:AXIS2:OPSTAT 0   % Отменить подписку
\end{gverb}


\setlength{\parskip}{0em}
\setlength{\parindent}{0em}
\setlength{\leftskip}{8mm}

\subsection{Перечень комманд NCPI}
\setlength{\parskip}{0em}

\subsubsection*{NOT:SYSTem:STATus}
	Уведомление о готовности системы. \\
	Тип: изменение состояния. \\ 

	Система считается готовой, если ни одно из устройств не сигнализирует об ошибке. \\
	Параметр:\\
	0 – Готов\\
	1 – Не готов

\subsubsection*{NOT:AXIS\#:STATus}
	Уведомление о готовности оси.\\ 
	Тип: изменение состояния.\\
	
	Ось считается готовой, если ни одно из устройств оси не сигнализирует об ошибке. \\
	Параметр:\\
	0 – Готов\\
	1 – Не готов

\subsubsection*{NOT:DEV\#:STATus}
	Уведомление о готовности устройства. \\
	Тип: изменение состояния\\

	Параметр:\\
	0 – Готов\\
	1 – Не готов

\subsubsection*{NOT:AXIS\#:POSition}
	Уведомление о текущем положении по оси.\\
	Тип: непрерывный параметр 

\subsubsection*{NOT:AXIS\#:UPOSition}
	Уведомление о положении по оси в единицах unit.\\
	Тип: непрерывный параметр 

\subsubsection*{NOT:AXIS\#:OPSTATus}
	Уведомление о статусе операции.\\
	Тип: изменение состояния\\

	Параметр:\\
	0 – Нет операции\\
	1 – Задание на движение\\
	2 – Режим сканирования

\subsubsection*{NOT:AXIS\#:OPSTOPtype}
	Уведомление о типе прерывания последней операции.\\
	Тип: изменение состояния\\	

	Параметр:\\
	0 – операция начата\\
	1 – нормальное завершение операции\\
	2 – операция завершена командой стоп\\
	3 – аварийное завершение операции
	
\subsubsection*{NOT:AXIS\#:SCAN:POINT}
	Уведомление о достижении новой точки в режиме сканирования. \\
	Тип: событие с аргументом\\

	Возвращает номер точки в теле уведомления.
	
\subsubsection*{NOT:AXIS\#:SCAN:TRIGERROR}
	Уведомление об ошибки в системе синхронизации при ожидании обратного триггера. Момент генерации следующего триггера настал ранее, чем вернулся обратный.\\
    Тип: событие\\

\subsubsection*{NOT:AXIS\#:SCAN:LSWItch}
	Уведомление об изменении статуса концевых датчиков.\\
	Тип: изменение состояния\\

	Параметр:\\
	0 – нет срабатывания\\
	1 - левый\\
	2 - правый\\
	10 – выдается при срабатывании обоих концевиков (вероятна ошибка)


\newpage
\section{Examples}
\subsection{Example of script for product (python3)}

\begin{lstlisting}[style=python]
#!/usr/bin/env python3

import sys
import time
import socket
import threading

HOST = '192.168.1.224' # ip address
SCPI_PORT = 5025 
NCPI_PORT = 5026

s = socket.socket(socket.AF_INET, socket.SOCK_STREAM)
n = socket.socket(socket.AF_INET, socket.SOCK_STREAM)
s.connect((HOST, SCPI_PORT))
n.connect((HOST, NCPI_PORT))

# request identifier
s.send("AXIS0:STAT:IDN?\r\n".encode("utf-8"))
answer = s.recv(1024).decode("utf-8")
print("axis0 idn is {}".format(answer))

# set speed : 1 unit per second
s.send("AXIS0:USPE 1\r\n".encode("utf-8"))

# set acceleration : 2 seconds for acceleration to nominal speed 
s.send("AXIS0:ACCEL 2000\r\n".encode("utf-8"))

# absolute move to coordinate 4u 
print("move to 4u")
s.send("AXIS0:UMOV:ABS 4\r\n".encode("utf-8"))

time.sleep(5)

# absolute move to coordinate 0u
print("move to 0u")
s.send("AXIS0:UMOV:ABS 0\r\n".encode("utf-8"))
\end{lstlisting}

\newpage
\subsection{Example of script for product using notification system (python3)}
\begin{lstlisting}[style=python]
#!/usr/bin/env python3

import sys
import time
import socket
import threading

HOST = '192.168.1.224' # ip address
SCPI_PORT = 5025
NCPI_PORT = 5026

s = socket.socket(socket.AF_INET, socket.SOCK_STREAM)
n = socket.socket(socket.AF_INET, socket.SOCK_STREAM)
s.connect((HOST, SCPI_PORT))
n.connect((HOST, NCPI_PORT))

s.send("AXIS0:STAT:IDN?\r\n".encode("utf-8"))
answer = s.recv(1024).decode("utf-8")
print("axis0 idn is {}".format(answer))

print("enable notification")
n.send("NOT:AXIS0:UPOS SMOOTH, 0.1\r\n".encode("utf-8"))
n.send("NOT:AXIS0:OPSTAT 1\r\n".encode("utf-8"))

cancel = False
def foo():
  n.settimeout(0.5)
  while cancel is False:
    try:
      message = n.recv(1024).decode("utf-8")
    except:
      pass
    sys.stdout.write("notification: {}".format(message))

thr = threading.Thread(target=foo)
thr.start()

print("move to 4u")
s.send("AXIS0:UMOV:ABS 4\r\n".encode("utf-8"))

time.sleep(5)

print("move to 0u")
s.send("AXIS0:UMOV:ABS 0\r\n".encode("utf-8"))

time.sleep(5)
cancel = True
\end{lstlisting}
 
\newpage
\subsection{Example of script for product, scanning scenario (python3)}
\begin{lstlisting}[style=python]
#!/usr/bin/env python3
​
import sys
import time
import socket
import threading
​
HOST = '192.168.1.224' # ip address
SCPI_PORT = 5025 
NCPI_PORT = 5026
​
s = socket.socket(socket.AF_INET, socket.SOCK_STREAM)
n = socket.socket(socket.AF_INET, socket.SOCK_STREAM)
s.connect((HOST, SCPI_PORT))
n.connect((HOST, NCPI_PORT))
​
print("enable notifications")
n.send("NOT:AXIS0:OPSTAT 1\r\n".encode("utf-8"))
n.send("NOT:AXIS0:SCAN:POINT\r\n".encode("utf-8")) 
​
cancel = False
def foo():
  n.settimeout(0.5)
  while cancel is False:
    try:
      message = n.recv(1024).decode("utf-8")
    except:
      pass
    sys.stdout.write("notification: {}".format(message))

thr = threading.Thread(target=foo)
thr.start()

print("set scan points number to 10")
s.send("AXIS0:SCAN:POINTS 10\r\n".encode("utf-8"))
​
print("set scan distance to 10u")
s.send("AXIS0:SCAN:UMOVE 10\r\n".encode("utf-8"))
​
print("activate scanning mode")
s.send("AXIS0:SCAN:COMPSTART\r\n".encode("utf-8"))
​
print("move on 10u")
s.send("AXIS0:UMOV 10\r\n".encode("utf-8"))
# while moving, notifications on points passed are sent to NCPI_PORT 
​
time.sleep(5)
cancel = True
\end{lstlisting}

\newpage
\subsection{Example of script for product, manual trigger (python3)}
\begin{lstlisting}[style=python]
#!/usr/bin/env python3
​
import sys
import time
import socket
import threading
​
HOST = '192.168.1.224' # ip address
SCPI_PORT = 5025 
​
s = socket.socket(socket.AF_INET, socket.SOCK_STREAM)
s.connect((HOST, SCPI_PORT))

# Manual trigger mode is used to generate trigger manually 
# using AXIS:TRIGGER command. It can be usefull
# when debugging connection with measuring device.
​
print("set manual trigger mode")
s.send("AXIS0:MANTRIG 1\r\n".encode("utf-8"))
​
print("send trigger")
s.send("AXIS0:TRIGGER\r\n".encode("utf-8"))
​
# waiting
time.sleep(1)

# TRIGRETTIME requests time passed between
# trigger generated and reverse trigger recieved.
# It is usefull for scanning setup.
​
s.send("AXIS0:TRIGRETTIME?\r\n".encode("utf-8"))
answer = s.recv(1024).decode("utf-8")
print("time passed: {}".format(answer))

\end{lstlisting}

\newpage
\section{Typical problems}
\subsubsection*{Connection is not established}
Make sure that the control unit is avalible in local network. 
Check the hardware connection. 
Ping the positioner.
Make sure that the firewall on your PC does not block the connection.
Restart the control unit if needed.

\subsubsection*{Product does not react on control commands}
Check the format and text of command. Make sure that the command is terminated with line end symbol LF.

\subsubsection*{Product answers the commands, but does not move}
Check the SCPI error stack.
Check system and axes status using commands:
\setlength{\parskip}{1em}

\begin{gverb}
SYST:STAT?
AXIS\#:STAT?          
\end{gverb}

Please contact the manufacturer in case of errors.
\setlength{\parskip}{0em}

\subsubsection*{Axis moves in the same direction regardless of the given direction of movement.}
Check UNSAFE movement commands.
Make sure the axis limits are set correctly and that the current position is not down.
Reconfigure zero and limits if needed.

\subsubsection*{No notifications comes in continious scanning mode}
Make sure the axis supports continious scanning mode
Check that the trigger circuits are installed correctly. 
Check that scan setting are set correctly.

\end{document}

