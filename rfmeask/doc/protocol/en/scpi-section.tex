\section{SCPI commands}
\setlength{\parskip}{0em}
\setlength{\parindent}{0em}
\setlength{\leftskip}{8mm}

\subsection{Protocol and error stack}
Protocol version and error control. 
\subsubsection*{SYSTem:VERSion?}
	Get SCPI protocol version.
\subsubsection*{SYSTem:ERRor[:NEXT]?}
	Get next error message from stack.
\subsubsection*{SYSTem:ERRor:COUNt?}
	Get error stack length.
\subsubsection*{*CLS} 
	Clean error stack.

\subsection{Other standart functions}
\subsubsection*{*ESE}
	Ignored.
\subsubsection*{*ESE?}	
	Ignored. (return: 1)
\subsubsection*{*ESR?}
	Ignored. (return: 1)
\subsubsection*{*OPC}
	Ignored.
\subsubsection*{*OPC?}
	Ignored. (return: 1)
\subsubsection*{*RST}
	Ignored.
\subsubsection*{*SRE}
	Ignored.
\subsubsection*{*SRE?}
	Ignored. (return: 1)
\subsubsection*{*STB?}
	Ignored. (return: 1)
\subsubsection*{*WAI}
	Ignored.


\subsection{Positioner info}
Invariable parameters of the positioner, its devices and axes info.

\subsubsection*{*IDN?}
	Get positioner ID. 

\subsubsection*{SYSTem:DEVSTOTal?}
	Get number of devices registered.

\subsubsection*{SYSTem:axes_total?}
	Get number of axes regist.


\subsection{System commands}
Commands which affect all the axes and devices of system and system as a whole.

\subsubsection*{SYSTem:PRESet}
	Reset all the devices to default state, reset all settings.
	
\subsubsection*{SYSTem:IPADDR}
	Set IPv4 adress of positioner. Commas are used instead of points. Example:\\

	\begin{gverb}
	SYST:IPADDR 192,168,1,42
	\end{gverb}

\subsubsection*{SYSTem:STOP}
	Stop all axes.

\subsubsection*{SYSTem:POWOFF}
	Disable power of all axes drivers.\\
	\emph{* For individual products only. For most products, power management is not required.}

\subsubsection*{SYSTem:STATus?}
	Get system ready state.\\
	Answer:\\
	0 -- System is ready.\\
	1 -- System is not ready.

\subsection{Devices commands}
Information on the devices.

\subsubsection*{DEV\#:IDN?}
	Get device ID.

\subsubsection*{DEV\#:PRESET}
	Reset device parameters to default.

\newpage

\subsubsection*{DEV\#:STATus?}
	Get device ready state\\
	Answer:\\
	0 -- Device is ready.\\
	1 -- Device is not ready.
	
\subsubsection*{DEV\#:ALM?}
	Request alarm code of the device in case of its failure.\\
	\emph{* Answer depends on device type.}


\subsection{Axis commands. Compatibility}
Depending on functional purpose of the axis and on servo drive type the axis 
can support additional functions.

\subsubsection*{AXIS\#:COMPat:SCAN?}
	Ask if axis has synchronous scanning subsystem.

\subsection{Axis commands. Info and state}
\subsubsection*{AXIS\#:STATus:IDN?}
	Get axis ID.

\subsubsection*{AXIS\#:STATus:DEVS?}
	Get axis devices codes.

\subsubsection*{AXIS\#:STATus:POSition?}
	Get position in encoder pulses.

\subsubsection*{AXIS\#:STATus:UPOSition?}
	Get position in units.

\subsubsection*{AXIS\#:SETTINGS:RATIO?}
	Get number of encoder pulses per unit.

\subsubsection*{AXIS\#:SETTINGS:DEFSPEed?}
	Get default speed in revolutions per minutes.

\subsubsection*{AXIS\#:SETTINGS:DEFACCel?}
	Get default acceleration time in milliseconds.

\subsubsection*{AXIS\#:SETTINGS:MAXSPEed?}
	Get maximum speed in revolutions per minute.

\subsubsection*{AXIS\#:SETTINGS:MINAccel?}
	Get minimum acceleration time in milliseconds.

\pagebreak

\subsubsection*{AXIS\#:STATus[:STATus]?}
	Get axis ready state.\\
	Answer:\\
	0 –- Axis is ready.\\
	1 –- Axis is not ready.

\subsubsection*{AXIS\#:STATus:LSWItch?}
	Get limit switches state.\\
	Answer:\\
	0 -- no operation, \\
	1 -- left, \\
	2 -- right, \\
	10 -- both limit switches are triggered (likely an error).\\
	\emph{* Only for products with limit switches. 
	In case they are abscent 0 is always returned}

\subsubsection*{AXIS\#:STATus:OPcode?}
	Get code of the current operation.\\
	Answer:\\
	0 -- no operation,\\
	1 --  moving is in progress,\\
	-1 -- axis is in initialization state.

\subsection{Axis commands. System functions}
\subsubsection*{AXIS\#:PRESET}
	Set all the axes to default.

\subsubsection*{AXIS\#:SON}
	[Remove brake] and power up engine.\\
	\emph{* For individual products only. 
	For most products, power management is not required.}

\subsubsection*{AXIS\#:SOFF}
	Remove power from engine, [set brake].\\
	\emph{* For individual products only. 
	For most products, power management is not required.}

\subsection{Axis commands. Software limits, zero set}
Unlike other settings, zero and move limits settings do not reset when powering off. 

\subsubsection*{AXIS\#:SETTINGS:UBACKLIMit <arg>}
\subsubsection*{AXIS\#:SETTINGS:UBACKLIMit?}
	Set/get negative move limit.

\subsubsection*{AXIS\#:SETTINGS:UFORWLIMit <arg>}
\subsubsection*{AXIS\#:SETTINGS:UFORWLIMit?}
	Set/get positive move limit.

\subsubsection*{AXIS\#:SETTINGS:ULIMITS <back>, <forw>}
	Set both move limits; \emph{back} must be lesser then \emph{forw}.

\subsubsection*{AXIS\#:SETZERo}
	Define current axis position as zero.
	
\subsection{Axis commands. Movement}

\subsubsection*{AXIS\#:SPEeed <rpm>}
\subsubsection*{AXIS\#:SPEeed?}
	Set/get speed in revolutions per minute.

\subsubsection*{AXIS\#:USPEeed <ups>}
\subsubsection*{AXIS\#:USPEeed?}
	Set/get speed in unit per second.

\subsubsection*{AXIS\#:ACCel <ms>}
\subsubsection*{AXIS\#:ACCel?}
	Set/get acceleration/break time interval
	(for commands of MOV and JOG type).\\
	<ms> -- time interval in milliseconds. 

\subsubsection*{AXIS\#:MOVe[:RELative] <distance>}
	Move a distance (relative movement, in encoder pulses).\\
	<distance> -- number of pulses. Sign defines movement direction.

\subsubsection*{AXIS\#:MOVe:ABSolute <position>}
	Move to position (absolute movement, in encoder pulses).\\
	<position> -- coordinate in pulses.

\subsubsection*{AXIS\#:UMOVe[:RELative] <distance>}
	Move a distance (relative movement, in units).\\
	<distance> -- distance in units. Sign defines movement direction.

\subsubsection*{AXIS\#:UMOVe:ABSolute <position>}
	Move to position (absolute movement, in units).\\
	<position> -- coordinate in units.

\subsubsection*{AXIS\#:JOG <direction>}
	Begin movement with constant speed.\\
	<direction:> 1 -- forward, -1 -- backward.

\subsubsection*{AXIS\#:STOP}
	Stop axis.

\subsubsection*{AXIS\#:UNSAFE:MOVe <distance>}
	Move a distance ignoring limits (in encoder pulses).\\
	Command can be used for debugging. 
	Using it in regular mode is not recommended. 

\subsubsection*{AXIS\#:UNSAFE:UMOVe <distance>}
	Move a distance ignoring limits (in units).\\
	Command can be used for debugging. 
	Using it in regular mode is not recommended.


\subsection{Axis commands. Continious scanning}
Continious scanning system is avalible for axes with synchronization module.
Activating continious scanning mode makes synchronization device start sending
trigger signals when passing predefined points.\\

Also when passing predefined points or getting reverse trigger (see NOTRIGMODE)
notifications are generated (see NOT:AXIS\#:SCANPOINT). To get information on 
passing the points by client, you need to subscribe to the approptiate notifications. \\

The formula for scanning step is: STEP = UMOVE / (POINTS - 1)

\subsubsection*{AXIS\#:SCAN:MOVE <distance>}
\subsubsection*{AXIS\#:SCAN:MOVE?}
\subsubsection*{AXIS\#:SCAN:UMOVE <distance>}
\subsubsection*{AXIS\#:SCAN:UMOVE?}
	Set/get scanning zone.\\
	Distance here is distance, processed by the scanning algorithm in pulses or units.
	Hereinafter commands with U prefix work with units, commands without it -- with pulses.

\subsubsection*{AXIS\#:SCAN:FWRDzone <distance>}
\subsubsection*{AXIS\#:SCAN:FWRDzone?}
\subsubsection*{AXIS\#:SCAN:UFWRDzone <distance>}
\subsubsection*{AXIS\#:SCAN:UFWRDzone?}
	Set/get distance to the first point (in pulses/units). \\
	Distance here is distance, passed by the algorithm before scanning.
	
\subsubsection*{AXIS\#:SCAN:COMPSTART}
	Activate scanning mode. \\
	Activating scanning mode doesn't start movement, it only 
	turns on generation of triggers and notifications on passing points.

\subsubsection*{AXIS\#:SCAN:POINTS <npnts>}
\subsubsection*{AXIS\#:SCAN:POINTS?}
	Set/get scanning points number.\\
	<npnts> -- number of bypass point.

\subsubsection*{AXIS\#:SCAN:NOTRIGMODE <en>}
	Enable/disable waiting for the reverse trigger before the next point is notified.\\
	<en>:\\
	0 -- notification on reaching next point is send after getting reverse trigger (by default),\\
	1 -- waiting for reverse trigger disabled
	
\subsubsection*{AXIS\#:MANTRIGmode <en>}
	Enable/disable manual trigger mode.\\
	This mode is used to generate trigger manually with AXIS\#:TRIGGER command.\\
 	It can be usefull when debugging connection with measuring device.\\
	<en>:\\
	1 -- enable manual mode\\
	0 -- disable manual mode

\newpage

\subsubsection*{AXIS\#:TRIGGER}
	Generate trigger.\\
	To use this command, manual trigger mode must be enabled (see AXIS\#:MANTRIGmode).

\subsubsection*{AXIS\#:TRIGRETTIME?}
	Get reverse trigger return time (in milliseconds).\\
	This command requests time passed between trigger generated and reverse trigger recieved.\\ 
	It is usefull for scanning regime setup.