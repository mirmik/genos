\section{NCPI API Reference}
\setlength{\parskip}{1em}

\subsection{Introduction}
The NCPI system allows you to receive information about events occuring in a device without polling it.

When connected to server, the client must activate nessesary notification types (or "themes") using commands described in this section. 
Client can subscribe to any number of types simultaniously, but only one time per notification type. 
When the subscription is active, server generates notification messages with structure "Name:Themes [Answer]".
TCP guarantees message delivering.

NCPI has three message types:
\begin{itemize} 
\item simple notification,
\item status change notification, notification with argument,
\item smooth parameter notification.
\end{itemize}

\subsubsection*{Simple notification}
A simple event means a notification structure: SUBSYST:THEME and does not carry 
any information other then that the event is occured.
\begin{gverb}
NOT:SUBSYST:THEME 1         % Subscribe.
NOT:SUBSYST:THEME 0         % Unsubscribe.
                            %
> SUBSYST:THEME             % Notification format.
\end{gverb}

\subsubsection*{Status change notification, events with argument}
Status change notification occures when the corresponding variable is changed and passes the corresponding string, code or value. \\
\begin{gverb}
NOT:SUBSYST:THEME 1         % Subscribe.
NOT:SUBSYST:THEME 0         % Unsubscribe.
                            %
> SUBSYST:THEME <argument>  % Notification format.
\end{gverb}

\subsubsection*{Smooth parameter change notification}
Such notification pass information about continious parameters that are smoothly changing over time (for example, current position). There are two notification strategies:

\setlength{\parskip}{3em}
1. Notification with time discrete (in milliseconds). Discrete specifies the minimum time interval between notifications.
It is important to note that the real intervals of time exceed the preset interval, since the event is generated at the moment of receiving the response to the request 
of the specified parameter by the controlling system. 
This ensures maximum accuracy for the time the notification received, but does not allow to confidently calculate the time of the next notification.
Besides, the minimum interval for the system exists and messages can't be send with time interval shorter (usually this interval equals a few tens of milliseconds). 
\setlength{\parskip}{1em}
\begin{gverb}
	NOT:SUBSYST:THEME TIMERED,1000  % Subscribe with 1 second interval.
	NOT:SUBSYST:THEME 0             % Unsubscribe.
									%
	SUBSYST:THEME <argument>        % Notification format.
\end{gverb}

2. Notification when parameter is changed to a preset value.
As in the case of time interval notifications, the real parameter change value can be larger then preset value.
\begin{gverb}
NOT:SUBSYST:THEME SMOOTH,5.34   % Subscribe on value delta = 5.34.
NOT:SUBSYST:THEME 0             % Unsubscribe.
                                %
SUBSYST:THEME <argument>        % Notification format.
\end{gverb}

\subsection{Example}
\begin{gverb}
NOT:AXIS2:OPSTAT 1   % Subscribe to axis 2 status change.
                     %
> AXIS2:OPSTAT 1     % Axis 2 begins movement.
> AXIS2:OPSTAT 0     % Axis 2 ends movement
                     %
NOT:AXIS2:OPSTAT 0   % Unsubscribe.
\end{gverb}

\setlength{\parskip}{0em}
\setlength{\parindent}{0em}
\setlength{\leftskip}{8mm}

\subsection{NCPI commands}
\setlength{\parskip}{0em}

\subsubsection*{NOT:SYSTem:STATus}
	System ready notification.\\
	Type: status change. \\

	System is ready when none of the devices signals any error. \\
	Parameter:\\
	0 -- ready, \\
	1 -- not ready.

\subsubsection*{NOT:AXIS\#:STATus}
	Axis ready notification. \\
	Type: status change. \\

	Axis is ready when none of the devices signals any error. \\
	Parameter:\\
	0 -- ready, \\
	1 -- not ready.

\subsubsection*{NOT:DEV\#:STATus}
	Device ready notification. \\
	Type: status change. \\

	Parameter:\\
	0 - ready, \\
	1 - not ready

\subsubsection*{NOT:AXIS\#:POSition}
	Axis position notification. \\
	Type: continious parameter.

\subsubsection*{NOT:AXIS\#:UPOSition}
	Axis position notification (in units). \\
	Type: continious parameter.

\subsubsection*{NOT:AXIS\#:OPSTATus}
	Status of operation notification. \\
	Type: status change. \\

	Parameter:\\
	0 - No operation, \\
	1 - Task for movement, \\
	2 - Scan Mode.

\subsubsection*{NOT:AXIS\#:OPSTOPtype}
	Notification on interruption of last operation (carries the interuption type). \\
	Type: status change. \\

	Parameter:\\
	0 - operation started, \\
	1 - normal completion of the operation, \\
	2 - operation completed with stop command, \\
	3 - emergency termination of the operation.
	
\subsubsection*{NOT:AXIS\#:SCAN:POINT}
	Notification on reaching a new point in scan mode.  \\
	Type: notification with argument.\\

	Returns the point number in the body of the notification.

\subsubsection*{NOT:AXIS\#:SCAN:TRIGERROR}
	Notification on synchronisation system error during waiting for reverse trigger 
	(The new trigger is generated before the previous one is returned).\\
	Type: event.

\newpage

\subsubsection*{NOT:AXIS\#:SCAN:LSWItch}
	Limit switches status change notification.\\
	Type: status change.\\

	Parameter:\\
	0 - no operation, \\
	1 - left, \\
	2 - right, \\
	10 - both limit switches are triggered (likely an error).

